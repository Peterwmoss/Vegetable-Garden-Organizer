\newpage
\section{Dokumentation}

\subsection{Overordnede opbygning}

\subsection{XML}

\subsection{Opret et nyt planteområde}
Den overordnede aktivitet til at oprette et nyt planteområde \texttt{CreateGardenActivity.kt} starter med at indsætte \texttt{SpecifyLocationFragment.kt}, som lader brugeren vælge et område til deres planter.

\subsubsection{Vælg område}
\texttt{fragment\_specify\_location.xml} indeholder et \texttt{TextView} og tre \texttt{ImageButtons} hver med tilhørende \texttt{TextView}, da der er tre forskellige typer af områder, som kan have indflydelse på planters pasning. Hver \texttt{ImageButton} har en \texttt{drawable} som src, der består af en \texttt{selector} med hver state for billedet. Der er således defineret et billede for tryk, fokusering og normal tilstand af hver knap.

Hver af knapperne har en \texttt{listener}. Når brugeren har valgt en lokation, sørger den gældende \texttt{listener} for at skifte fragment, og sende info om den valgte lokation med i fragmentets konstruktør i form af et \texttt{enum}.

\subsubsection{Byg layoutet}

\subsubsection{Vælg planter}
